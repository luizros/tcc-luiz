\begin{resumo}



Avanços em inteligência artificial, particularmente em redes neurais profundas (DNNs) e aprendizado por reforço (RL), têm impulsionado a modelagem comportamental de sistemas multiagente em robótica competitiva, demandando avaliações comparativas de arquiteturas para cenários dinâmicos, como a \textit{Small Size League} (SSL) da RoboCup. Este trabalho propõe uma análise sistemática de modelos de aprendizado de máquina para prever comportamentos de robôs futebolísticos, criando uma base metodológica para o desenvolvimento de modelos em cenários semelhantes. A metodologia baseia-se no uso de dados históricos de partidas para treinar e validar modelos preditivos, além de comparar as diferentes arquiteturas empregadas, avaliando sua eficácia na previsão de comportamentos, padrões táticos e interações colaborativas entre agentes. Os resultados empíricos buscam identificar a abordagem mais robusta para integração em sistemas autônomos em tempo real, contribuindo com um \textit{framework} replicável para a modelagem de oponentes em competições de robótica e estabelecendo bases para a otimização estratégica em ambientes com restrições operacionais e computacionais.


 \vspace{\onelineskip}
    
 \noindent
 \textbf{Palavras-chaves}: Aprendizado de Máquina. Deep Learning. Filtro de Kalman. Estados. Robôs Futebolistas. Small Size League.
\end{resumo}
