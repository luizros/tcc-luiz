% \chapter[Referencial Tecnológico]{Referencial Tecnológico}

% \section{Jupyter Notebook} O Jupyter Notebook é um ambiente interativo de código aberto para criação e compartilhamento de documentos que combinam código executável (e.g., Python, R), visualizações de dados, texto explicativo e equações. Sua estrutura baseada em células facilita a exploração iterativa de dados, prototipagem de algoritmos e documentação de fluxos de trabalho científicos.

% \section{PyTorch} O PyTorch é uma biblioteca de aprendizado profundo baseada em tensores, destacando-se pela flexibilidade e grafos computacionais dinâmicos. Desenvolvido pelo Facebook AI Research (FAIR), permite a construção e treinamento de redes neurais, com suporte a aceleração de cálculos por GPU/TPU.

% \section{Scipy, Matplotlib e Pandas} O Scipy oferece módulos para álgebra linear, otimização e processamento de sinais. O Matplotlib é utilizado para visualização de dados 2D/3D, enquanto o Pandas proporciona estruturas de dados eficientes para manipulação de dados tabulares, integrando-se a fluxos de trabalho de análise e machine learning.

% \section{SSL-GO-TOOLS} O ssl-go-tool é um conjunto de pacotes em Go que facilita tarefas na RoboCup Small Size League (SSL). Oferece funcionalidades para leitura, escrita, envio, recebimento e análise de mensagens do SSL-Vision e SSL-GameController, sendo utilizado também para extrair dados de logs de partidas e rodar partidas anteriores através de um log player.