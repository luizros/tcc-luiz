\begin{resumo}[Abstract]
 \begin{otherlanguage*}{english}
Advances in artificial intelligence, particularly in deep neural networks (DNNs) and reinforcement learning (RL), have driven the behavioral modeling of multi-agent systems in competitive robotics, requiring comparative evaluations of architectures for dynamic scenarios such as the Small Size League (SSL) of RoboCup. This work proposes a systematic analysis of machine learning models to predict the behavior of soccer-playing robots, creating a methodological foundation for developing models in similar scenarios. The methodology is based on using historical match data to train and validate predictive models, as well as comparing different employed architectures, assessing their effectiveness in predicting behavior, tactical patterns, and collaborative interactions between agents. The empirical results aim to identify the most robust approach for integration into real-time autonomous systems, contributing to a replicable framework for opponent modeling in robotics competitions and establishing foundations for strategic optimization in environments with operational and computational constraints.

   \vspace{\onelineskip}
 
   \noindent 
   \textbf{Key-words}: Machine Learning. Deep Learning. Kalman Filter. States Soccer Robots. Small Size League.
.
 \end{otherlanguage*}
\end{resumo}
